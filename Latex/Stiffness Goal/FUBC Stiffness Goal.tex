\documentclass[a4paper]{article}

\newcommand\tab[1][1cm]{\hspace*{#1}}
\usepackage{float}
\restylefloat{table}

\usepackage[english]{babel}
\usepackage[utf8]{inputenc}
\usepackage{amsmath}
\numberwithin{equation}{section}
\usepackage{graphicx}
\usepackage[colorinlistoftodos]{todonotes}
\usepackage[toc,page]{appendix}
\usepackage[framed,numbered,autolinebreaks,useliterate]{mcode}
\usepackage[margin=1in]{geometry}
\usepackage[parfill]{parskip}

\title{Formula UBC Torsional Stiffness Goal}

\author{Scott Fjordbotten}

\date{\today}

\begin{document}
	\maketitle 
	\begin{figure*}[h]
		\includegraphics[width=\textwidth]{"Figures/FUBClogoInvert".png}
	\end{figure*}

	\newpage
	\tableofcontents
	\listoffigures

\newpage
	
\section{Project Description} \label{Project Description}
The torsional stiffness of the FUBC chassis is an important design parameter as it determines how the vehicle will handle and the degree to which the suspension can be tuned to improve vehicle dynamics. There are a variety of rules of thumb and suggested values for torsional chassis stiffness available in literature on the topic of chassis design and vehicle dynamics, however, it is desirable to be able to complete our own analysis to either validate these rules of thumb or create our own torsional stiffness goal.

The purpose of this research project will be to identify the roll torsional chassis stiffness plays in vehicle dynamics, write a MATLAB script to allow analysis of the effects of chassis stiffness given a set of vehicle parameters, and ultimately provide a means for setting justified goals for chassis torsional stiffness during the design phase of subsequent FUBC cars.

\subsection{Previous Consideration of Torsional Stiffness During Chassis Design} \label{Previous Consideration of Torsional Stiffness During Chassis Design}
The importance of torsional stiffness and the analysis of torsional stiffness during design has been considered to a limited extent in the past. As far as I can tell, torsional stiffness goals have been set based on approximate values found in papers released by other Formula Student teams, and comparison to previous FUBC designs that have been deemed to be adequately stiff “by feel” while driving and tuning suspension.

The torsional stiffness of chassis design revisions are determined through FEA throughout the design process. In 2016, a torsional stiffness jig for physical testing was constructed. However, due to inadequate knowledge transfer from team members who had worked on the design of the jig over the past 3 or 4 years and inadequate design review, the jig did not perform as desired. Physical testing data was taken in 2016, but the test setup was far from ideal and the accuracy of the data was questionable; fortunately, the physical test data matched the FEA data reasonably well. In the future, a new torsional stiffness jig should be designed and constructed (adhering to a more formal and structured design and review process) to allow FEA results to be validated via physical testing.

\subsection{Background Research and Performance Indicators} \label{Background Research and Performance Indicators}
When a vehicle turns, the lateral acceleration of the turn coupled with height of the vehicle’s mass components above the suspension roll center exert a moment about the lateral axis of the vehicle. This moment is reacted by vertical forces at the tires; the distribution of the reacting forces between the front and rear tires is called the lateral load distribution and dictates how the vehicle handles through the corner.

Ideally, the front and rear roll stiffness of the vehicle’s suspension can be tuned to achieve the desired lateral load transfer distribution. In order for suspension roll stiffness to be the vehicle parameter dictating lateral load transfer distribution, the chassis must be adequately stiff in torsion. To demonstrate the importance of chassis stiffness, consider the front and rear suspension to be torsional springs and the chassis to be either a perfectly ridged or perfectly flexible rod connecting the suspension springs. 

If Mlat is the moment experienced by the vehicle due to lateral acceleration, the moment exerted by the suspension must equal the moment caused by lateral acceleration for the vehicle to be static equilibrium:

$$M_{lat} = M_{front} + M_{rear}$$

Where $M_{front}$ is the moment exerted by the front torsional spring and $M_{rear}$ is the moment exerted by the rear torsional spring and the moment exerted by each spring is proportional to the stiffness of the springs:

$$M_{front} = K_{front}\theta_{front}$$
$$M_{rear} = K_{rear}\theta_{rear}$$

In the case of a perfectly ridged rod, the angular rotation of each torsional spring must be the same:

$$\theta_{front} = \theta_{rear}-\theta$$

The moment balance becomes:

$$M_{lat} = K_{front}\theta + K_{rear}\theta$$

And the proportion of the moment caused by lateral acceleration reacted by the front and rear of the vehicle is determined by the suspension stiffness as desired.

In the case of the perfectly flexible rod, the moment exerted by each torsional spring must be the same:

$$M_{front} = M_{rear} = M$$
\begin{center}
	or
\end{center}
$$K_{front}\theta_{front} = K_{rear}\theta_{rear}$$

The moment balance becomes:

$$M_{lat} = 2 \times M$$

And the proportion of the moment caused by lateral acceleration reacted by the front and rear of the vehicle does not change as suspension stiffness is varied.

In reality, a chassis is neither perfectly ridged nor perfectly flexible. The stiffness of the chassis is a function of the material used, geometry, and the size of the vehicle. For a given material, increasing the torsional stiffness of the chassis requires more efficient material use or more material (i.e. more mass); increasing the stiffness usually means increasing the mass of the chassis. As a low vehicle weight is desirable, it is important to determine what chassis stiffness is “stiff enough” so that vehicle mass is not increased unnecessarily.

While researching torsional stiffness goals, three different recommendations were discovered:

\begin{enumerate}
	\item Deakin et al. \cite{Deakin} suggest that about 80\% of the difference in front to rear roll stiffness must result in a difference in front to rear lateral load transfer.
	\item Milliken and Milliken’s Race Car Vehicle Dynamics \cite{Milliken} states that the chassis stiffness can be approximately designed to be X times the total suspension roll stiffness, or X times the difference between front and rear suspension stiffness. X is said to be somewhere in the range of 3 - 5 times.
	\item In Fundamentals of Automobile Body Structure Design \cite{Malen}, Malen suggests yet another rule of thumb. He states that suspension is designed assuming a ridged chassis and that to make this assumption valid, chassis stiffness should be approximately 10 times the total suspension stiffness.
\end{enumerate}

Due to the conflicting suggestions, this project will determine which of the suggestions are feasible for an FSEA car and determine if one of the feasible suggestions will be followed or if another should be developed.

\section{Feasibility} \label{Feasibility}
To evaluate stiffness feasibility, a MATLAB package was developed to solve forces in minimally stable, statically determinate 3D truss structures in which all members experience only tensile or compressive forces. A macro was also written to allow such truss structures to be extracted from 3D sketches in SolidWorks.

The initial intension of the package was to analyze a simplified version of the FUBC chassis, however it was soon determined that the chassis could not be analyzed in this way. Instead, the chassis was simplified to a square rectangular prism with length equal to the vehicle’s wheelbase and square side length ranging from the width of the front bulkhead to the designed track width for the FUBC 2017 car (Figure 1). The tubes in the prism were considered to be 1” OD (most tubes used in the 2016 FUBC care were this diameter) 0.057” thick tubes. This thickness is the length weighted average thickness of the tubes used in the 2016 FUBC car. 

% Tube approximation figures
\begin{figure}[h]
	\begin{center}
		\includegraphics[width=0.75\textwidth]{"Figures/SWTubeBox".png}
	\end{center}
	
	\caption{Solidworks sketch of simplified chassis model as a square rectangular prism}
	\label{fig:SWTubeBox}
\end{figure}

\begin{figure}[h]
	\begin{center}
		\includegraphics[width=0.75\textwidth]{"Figures/MatlabTubeBox".png}
	\end{center}
	
	\caption{Simplified chassis model as a square rectangular prism imported into MATLAB}
	\label{fig:MatlabTubeBox}
\end{figure}

A torque was applied to one end of the structure, while the other was fixed and energy methods were used to determine the angle of rotation of the structure:

\begin{equation}
\frac{1}{2}T\theta = \Sigma(tube strain energy)=\Sigma\frac{\sigma_i\epsilon_i}{2}A_i L_i = \Sigma\frac{\sigma_i^2}{2E_i}A_i L_i
\end{equation}

\begin{equation}
\theta = \frac{2}{T}\Sigma\frac{\sigma_i^2}{E_i}A_i L_i
\end{equation}

Where T is the applied torque, subscript i denotes the ith tube, $\sigma$ is the stress a tube, $\epsilon$ is the strain in a tube, $E$ is the young’s modulus for a tube, $A$ is the cross sectional area of the tube, and $L$ is the length of a tube.

Now, stiffness can calculated as:
\begin{equation}
K=\frac{T}{\theta} = \Sigma\frac{\sigma_i^2}{E_i} A_i L_i
\end{equation}

Analysis was run and fourth order polynomial fit to the data was created to determine the approximate dimensions required to meet each suggested chassis stiffness. The suggestions were then evaluated based on the spatial constraints of the car. The preliminary 2017 FUBC vehicle parameters used are summarized in Table \ref{table:projVehicleParams}, and the results from analysis are summarized in Table \ref{table:Feasibility analysis results}. Due to the requirement of lateral load transfer distribution for the Deakin et al. suggestion, it was not considered at this stage.


\begin{table}[H]
	\begin{center}
		\begin{tabular}{|c|c|}
			\hline
			Vehilce Property & Value [Unit]\\
			\hline
			Wheelbase & 1.54 [m]\\
			\hline
			Front Track Width &1.22 [m]\\
			\hline
			Rear Track Width & 1.19 [m]\\
			\hline
			Bulkhead Width & 0.33 [m]\\
			\hline
			Widest Section (Main roll hoop) & 0.67 [m]\\
			\hline
		\end{tabular}
		\caption{Projected 2017 FUBC vehicle parameters}
		\label{table:projVehicleParams}
	\end{center}
\end{table}

\begin{table}[H]
	\begin{center}
		\begin{tabular}{|c|c|c|}
			\hline
			Chassis Stiffness Suggestion & Stiffness Target [Nm/deg] & Square Side Length [m]\\
			\hline
			3X Roll Stiffness & 2340 & 0.727\\
			\hline
			5X Roll Stiffness & 3900 & 0.85\\
			\hline
			10X Roll Stiffness & 7800 & 1.07\\
			\hline
		\end{tabular}
		\caption{Feasibility analysis results}
		\label{table:Feasibility analysis results}
	\end{center}
\end{table}

The results immediately ruled out the Malen suggestion of 10X the roll stiffness as this is projected to require chassis dimensions nearly as large as the projected track width; this is clearly not feasible and will not be considered further.  Milliken and Milliken’s suggestion of 3-5 times the roll stiffness seems more reasonable; although the predicted chassis dimensions are larger than our projected design, it is important to acknowledge that there will be some error in this overly simple model of the chassis. Thus, the 3-5 times roll stiffness suggestion will be given further consideration in the next section.

\section{Model and Script Development} \label{Model and Script Development}
In order to justify the rules of thumb for chassis stiffness, or to create a new way to set torsional stiffness goals at FUBC, the simple model described in section \ref{Background Research and Performance Indicators} to show the importance of torsional chassis stiffness was built upon. This model is based upon the model used by Eurenius et al. \cite{Chalmers} and Deakin et al. \cite{Deakin} and will be described in detail in section \ref{The Model}.

\subsection{The Model} \label{The Model}
Deakin et al. \cite{Deakin} developed a simple model for calculating the static forces present in the chassis under steady state conditions. This model considers the racing car to consist of two point masses, $m_f$ and $m_r$ for the front and rear respectively, connected by a torsional spring, $K_{ch}$, and suspension at each end of the vehicle represented by a roll stiffness, $K_{rollf}$ and $K_{rollr}$, Figure \ref{fig: DeakinModel}. 

This simple model was built upon by Eurenius et al. \cite{Chalmers}; the vehicle mass is divided further into a sprung component, $m_{sprung}$, located at the vehicles center of mass and two unsprung components, $m_{frontUnsprung}$ and $m_{rearUnsprung}$, representing the unsprung mass at the front and rear axles, respectively. The unsprung masses are assumed to be at the height of the front and rear wheel centers; further, the unsprung mass distribution is assumed to be symmetric about the longitudinal axis of the vehicle. This model is shown in Figure \ref{fig: ChalmersModel}.

% Model figures
\begin{figure}[h]
	\begin{minipage}{0.45\textwidth}
		\centering
		\includegraphics[width=0.95\textwidth]{"Figures/DeakinModel".png}
		\caption{Deakin et al. \cite{Deakin} simplified vehicle model}
		\label{fig: DeakinModel}
	\end{minipage}
	\hspace{0.1\textwidth}
	\begin{minipage}{0.45\textwidth}
		\includegraphics[width=0.95\textwidth]{"Figures/ChalmersModel".png}
		\caption{Eurenius et al. \cite{Chalmers} simplified vehicle model}
		\label{fig: ChalmersModel}
	\end{minipage}
\end{figure}

Both models assume a uniform chassis stiffness distribution along the length of the chassis. This allows Deakin et al. to divide the vehicle mass between the front and rear point masses based on the geometric location of the center of mass \cite{Deakin}. Similarly, Eurenius et al. divide the moment caused by lateral acceleration of the sprung mass between the front and rear suspension based on geometry \cite{Chalmers}.

Deakin et al. note that in reality neither the mass nor stiffness distribution of the chassis is uniform, so there will likely be some discrepancies between the simplified model and the actual vehicle. Further, compliances in the suspension, commonly referred to as the installation stiffness, reduce the chassis torsional stiffness as seen at the wheels. Installation stiffness should also be considered as possible cause of discrepancy between the idealized model and the real vehicle.

The model used by Eurenius et al. was adopted as is seems to take into account more of the vehicles mass distribution, which was expected to reduce the error in the idealized model.

Next the moment arms for each mass must be determined to determine the torque experienced by the suspension and chassis during steady state cornering. The method used is again based on the model used by Eurenius et al., in which the front and rear roll centers, wheel centers, and the height of the center of mass are used to calculate the moment arms. Since the roll centers for the front and rear suspension are known, the height of the unsprung masses (wheel centers) above their respective roll center is used as the moment arm for the unsprung masses. The moment arm for the sprung mass is determined by drawing a roll center line between the front and rear roll centers. The height of this line at the longitudinal position of the center of gravity can then be calculated and the moment arm is the difference between the height of the center of mass and the interpolated roll center height at the longitudinal position of the center of mass. This geometry is shown in Figure \ref{fig: MassRollGeometry} and the following moment arm equations.

\begin{figure}[h]
	\begin{center}
		\includegraphics[width=\textwidth]{"Figures/MassRollGeometry".png}
	\end{center}
	
	\caption{Mass and roll center geometry, from Eurenius et al. \cite{Chalmers}}
	\label{fig: MassRollGeometry}
\end{figure}

\begin{equation}
x = h - \frac{am+bn}{a+b} \text{\tab(Sprung mass moment arm)}
\end{equation}

\begin{equation}
\text{Front Unsprung Mass Moment Arm}=r_1-n
\end{equation}

\begin{equation}
\text{Rear Unsprung Mass Moment Arm}=r_2-m
\end{equation}

By choosing a lateral acceleration, $a_{lat}$, the moment exerted by each mass can be calculated:

\begin{equation}
M_{sprung} = a_{lat} m_{sprung} x
\label{equ: sprung}
\end{equation}

\begin{equation}
M_{FrontUnsprung} = a_{lat} m_{frontUnsprung} (r_1-n)
\label{equ: unsprung front}
\end{equation}

\begin{equation}
M_{RearUnsprung}=a_{lat} m_{rearUnsprung} (r_2-m)
\label{equ: unsprung rear}
\end{equation}

The moment due to the sprung mass can then be distributed between the front and rear based on vehicle geometry:

\begin{equation}
M_{frontSprung} = \frac{b}{a+b} M_{sprung}
\label{equ: sprung front}
\end{equation}

\begin{equation}
M_{rearSprung} = \frac{a}{a+b} M_{sprung}
\label{equ: sprung rear}
\end{equation}

Now the front and rear suspension springs will want to deflect angularly due to the applied moment and the stiffness of the chassis will apply moments to counteract any difference in suspension angular deflection. Using the resuts from equation \ref{equ: sprung} to \ref{equ: sprung rear}, equating the applied and resultant moments, and requiring the difference in suspension angular deflections to be equal to the chassis angular deflection results in the following system of equations:

\begin{equation}
	\begin{bmatrix}
		K_{front} & 0 & -K_{chassis}\\
		0 & K_{rear} & K_{chassis}\\
		1 & -1 & 1
	\end{bmatrix}
	\begin{bmatrix}
		\phi_{front}\\
		\phi_{rear}\\
		\phi_{chassis}\\
	\end{bmatrix}
	=
	\begin{bmatrix}
		M_{frontUnsprung} + \frac{b}{a+b}M_{sprung}\\
		M_{rearUnsprung} + \frac{a}{a+b}M_{sprung}\\
		0
	\end{bmatrix}
	=
	\begin{bmatrix}
		M_{front}\\
		M_{rear}\\
		0
	\end{bmatrix}
	\label{equ: model system}
\end{equation}

The above matrix equation can either be solve using matrix algebra and MATLAB, or by hand. The expressions for each angle of rotation are:

\begin{equation}
	\phi_{front} = \frac{M_{front} K_{front}+K_{chassis} (M_{front}+M_{rear})}{K_{front} K_{rear}+K_{chassis} (K_{front}+K_{rear})}	
	\label{equ: phi front}
\end{equation}

\begin{equation}
	\phi_{rear} = \frac{M_{rear} K_{front}+K_{chassis} (M_{front}+M_{rear})}{K_{front} K_{rear}+K_{chassis} (K_{front}+K_{rear})}
	\label{equ: phi rear}
\end{equation}

\begin{equation}
	\phi_{chassis} = \frac{M_{rear} K_{front}-M_{front}K_{rear}}{K_{front} K_{rear}+K_{chassis} (K_{front}+K_{rear})}
	\label{equ: phi chassis}
\end{equation}

The lateral load transfer, the load that shifts from the inside wheel to the outside wheel, at either the front or rear wheels is defined as:

\begin{equation}
	LT = \frac{P_{outside}-P_{inside}}{2}
	\label{equ: load transfer}
\end{equation}

Using moment balance at the front axle, the moment caused by the vertical tire forces must equal the moment exerted by the front suspension (Figure \ref{fig: load transfer}).

\begin{figure}[h]
	\begin{center}
		\includegraphics[width=\textwidth]{"Figures/loadTransfer".png}
	\end{center}
	
	\caption{Front axle moment balance. Adapted from Eurenius et al. \cite{Chalmers}}
	\label{fig: load transfer}
\end{figure}

$$\frac{P_{outside}-P_{inside}}{2}t_{front}=LT_{front}t_{front}=K_{front}\phi_{front}=M_{front}+K_{chassis}\phi_{chassis}$$

Where $t_{front}$ is the front track width. Solving for the load transfer at the front wheels:

\begin{equation}
	LT_{front}=\frac{M_{front}+K_{chassis}\phi_{chassis}}{t_{front}}
	\label{equ: load transfer front}
\end{equation}

Similarly for the rear wheels:

\begin{equation}
	LT_{rear}=\frac{M_{rear}-K_{chassis}\phi_{chassis}}{t_{rear}}
	\label{equ: load transfer rear}
\end{equation}

Now the load transfer distribution, reported as the percentage of the load transfer that occurs at the front or rear wheels, can be calculated using equations \ref{equ: load transfer front} to \ref{equ: load transfer rear}:

\begin{equation}
LLTD_{\%front} = \frac{LT_{front}}{LT_{front}+LT_{rear}}
\label{equ: load transfer dist front}
\end{equation}

\begin{equation}
LLTD_{\%rear} = \frac{LT_{rear}}{LT_{front}+LT_{rear}} = 1 - LLTD_{\%front}
\label{equ: load transfer dist rear}
\end{equation}

To determine chassis stiffness performance as described in Deakin et al., the difference in front and rear lateral load transfer distribution is compared to the difference in front and rear roll stiffness distribution. This metric has a maximum value of 1, corresponding to a ridged chassis, and is shown below:

\begin{equation}
Index=\frac{|LLTD_{\%front}-LLTD_{\%rear}|}{|RSD_{\%front}-RSD_{\%rear}|}
\label{equ: chassis index}
\end{equation}

Where RSD is the suspension roll stiffness distribution as a percentage.

\subsection{The Script} \label{The Script}
The model described in section \ref{The Model} was used to create a MATLAB script that, given vehicle parameters, could calculate the lateral toad transfer and lateral load transfer distribution difference per roll stiffness distribution difference index for a range of weight distributions, suspension roll stiffness distributions, and chassis stiffness’s.

The script was broken into two parts:
\begin{enumerate}
	\item A function calc\_lltd\_per\_rsd.m: given vehicle properties and ranges of chassis stiffness and roll stiffness distribution to consider, calculates the index presented at the end of section \ref{The Model} and both the front and rear load transferred for a range of roll stiffness distributions and chassis stiffness’s.
	\item A script LLTD.m: contains vehicle parameters and organizes the calling of calc\_lltd\_per\_rsd.m and the plotting of its results.
\end{enumerate}

The remainder of this section will explain how these two scripts function in more detail, what data was used to generate the plots presented in section \ref{Conclusion}, and how to use the scripts. The purpose of this section is to aid in understanding of the script for future use and development.

\subsubsection{calc\_lltd\_per\_rsd.m} \label{calc_lltd_per_rsd.m}
This function takes the sprung mass, front and rear unsprung masses, center of gravity height (h in Figure \ref{fig: MassRollGeometry}), center of gravity distance from rear axle (b in Figure \ref{fig: MassRollGeometry}), front and rear wheel radii ($r_1$ and $r_2$ in Figure \ref{fig: MassRollGeometry}), front and rear roll center heights (m and n in Figure \ref{fig: MassRollGeometry}), front and rear track widths ($t_{front}$ and $t_{rear}$ from equations \ref{equ: load transfer front} and \ref{equ: load transfer rear}), wheelbase ($a+b$ from Figure \ref{fig: MassRollGeometry}), total roll stiffness ($K_{front} + K_{rear}$ from equation \ref{equ: model system}), minimum and maximum front roll stiffness distribution, minimum and maximum chassis stiffness, and lateral acceleration as parameters. The return vaues are a vector with the values of chassis stiffness considered, a vector with the result of equation \ref{equ: chassis index} for each chassis stiffness considered, and matrices containing the front and rear load transfer for each chassis stiffness-Roll roll stiffness distribution pair considered.

The first section of the script sets up vectors to iterate through chassis stiffness and roll stiffness distributions as determined by the function input. Next, the dimensions a, b and x from Figure \ref{fig: MassRollGeometry} are calculated.

Equations \ref{equ: sprung}, \ref{equ: unsprung front}, \ref{equ: unsprung rear}, \ref{equ: sprung front}, and \ref{equ: sprung rear} are used to calculate the moment experienced at the front and rear axles. Then, for each chassis stiffness and roll stiffness distribution, equations \ref{equ: model system}, \ref{equ: load transfer front}, and \ref{equ: load transfer rear} are used to calculate the load transfer at the front and rear wheels.

Finally, equations \ref{equ: load transfer dist front} to \ref{equ: chassis index} are used to calculate the index measuring how much of the suspension roll stiffness distribution translates into lateral load transfer distribution. The chassis stiffness index value is an average over the first and last quarter of the roll stiffness distributions considered. This is an attempt to avoid the point where the index is singular (50-50 roll stiffness distribution), and to ignore the common crossover point in the data where mass distribution and roll stiffness distribution are opposite (see Section \ref{Conclusion}). The latter is an ideal case where mass distribution and roll stiffness distribution effectively cancel each other out, making the chassis stiffness in the idealized model inconsequential and telling us nothing about the chassis' impact on how effectively the suspension can be tuned.

The function in its entirety can be found in Appendix \ref{LLTD Scripts}.

\subsubsection{LLTD.m} \label{LLTD.m}
This script defines the variables used by calc\_lltf\_per\_rsd.m, defines what mass distributions will be considered, calls calc\_lltf\_per\_rsd.m, and interprets and plots the output from calc\_lltf\_per\_rsd.m.

Each mass distribution (determined my the centre of mass location) is passed, along with the other required vehicle parameters, to calc\_lltf\_per\_rsd.m. The chassis stiffness value where transferTarget percent of the difference in roll stiffness distribution is seen as lateral load transfer distribution is determined, and this along with the chassis index vs chassis stiffness is plotted in figure 1.

When the middle weight distribution value is reached, curves of lateral load distribution expressed as the percentage of the load transfer that occurs at the front axle vs the roll stiffness distribution (again expressed as percent front) are plotted. This is done for four chassis stiffness' in figure 2. This shows how LLTD changes with roll stiffness distribution and how this relationship changes with chassis stiffness. Next, curves of lateral load distribution expressed as the percentage of the load transfer that occurs at the front axle vs chassis stiffness are plotted. This is done for three roll stiffness distributions to show how chassis stiffness changes the lateral load distribution for a given roll stiffness distribution.

Last, legends and axis labels are added to the plots for completeness.

The output plots are presented and analyzed in section \ref{Conclusion} and the script can be found in its entirety in Appendix \ref{LLTD Scripts}.

\section{Conclusion} \label{Conclusion}

\newpage
\begin{thebibliography}{9}
	\bibitem{Chalmers}
	C. Andersson Eurenius, N. Danielsson, A. Khokar, E. Krane, M. Olofsson and J. Wass.
	\textit{Analysis of Composite Chassis}.
	Chalmers University of Technology, Göteborg, Sweden, 2013.
	
	\bibitem{Deakin}
	A. Deakin, D. Crolla, J. P. Ramirez and R. Hanley.
	\textit{The Effect of Chassis Stiffness on Race Car Handling Balance}.
	SAE Technical Paper, 2013.
	
	\bibitem{Malen}
	D. E. Malen.
	\textit{Fundamentals of Automobile Body Structure Design}.
	SAE International, 2011. 
	
	\bibitem{Milliken}
	W. F. Milliken and D. L. Milliken.
	\textit{Race Car Vehicle Dynamics}.
	SAE International, 1994. 
	
	\label{References}
\end{thebibliography}

\begin{appendices}
	\section{Truss FEA Scripts} \label{FEA Scripts}
	
	\section{LLTD Scripts} \label{LLTD Scripts}
\end{appendices}
\end{document}