\section{LLTD Scripts} \label{LLTD Scripts}
The model described in section \ref{The Model} was used to create a MATLAB script that, given vehicle parameters, could calculate the lateral toad transfer and proportion of roll stiffness distribution that translates into lateral load transfer distribution for a range of weight distributions, suspension roll stiffness distributions, and chassis stiffness’s.

The script was broken into two parts:
\begin{enumerate}
	\item A function calc\_lltd\_per\_rsd.m: given vehicle properties and ranges of chassis stiffness and roll stiffness distribution to consider, calculates the index presented at the end of section \ref{The Model} and both the front and rear load transferred for a range of roll stiffness distributions and chassis stiffness’s.
	\item A script LLTD.m: contains vehicle parameters and organizes the calling of calc\_lltd\_per\_rsd.m and the plotting of its results.
\end{enumerate}
		\subsection{calc\_lltd\_per\_rsd.m}
\begin{lstlisting}
function [k_chassis, dLLTD_by_dRSD, LLT_front, LLT_rear] =  calc_lltd_per_rsd( mSprung, mUnsprung_front, mUnsprung_rear, hCG, dist_cg_from_rear, rWheel_front, rWheel_rear, hRC_front, hRC_rear, trackWidth_front, trackWidth_rear, wheelBase, kRoll, kRoll_pctFrontMin, kRoll_pctFrontMax, kChassis_min, kChassis_max, latAcc)
% Calculates the lateral load distribution difference per roll
% stiffness distribution difference as a function of chassis stiffness.
% Parameters :
%   - mSprung : [kg] Sprung vehicle mass, including driver.
%   - mUnsprung_front : [kg] unsprung mass of front components
%   - mUnsprung_rear : [kg] unsprung mass of rear components
%   - hCG : [m] height of CG above the ground
%   - dist_cg_from_rear : [m] longitudinal distance from the rear axle to
%                         the CG.
%   - rWheel_front : [m] radius of the front wheels
%   - rWheel_rear : [m] radius of the rear wheels
%   - hRC_front : [m] height of the front roll center above the ground
%   - hRC_rear : [m] height of the rear roll center above the ground
%   - trackWidth_front : [m] track width at the front wheels
%   - trackWidth_rear : [m] track width at the rear wheels
%   - wheelBase : [m] wheel base of the vehicle
%   - kRoll : [Nm/deg] total roll stiffness (sum of front and rear roll
%             stiffness
%   - kRoll_pctFrontMin: [unitless] the minimum fraction of total roll
%                        stiffness to consider as front roll stiffness
%   - kRoll_pctFrontMax: [unitless] the maximum fraction of total roll
%                        stiffness to consider as front roll stiffness
%   - kChassis_min : [Nm/deg] minimum chassis stiffness to consider
%   - kChassis_max : [Nm/deg] maximum chassis stiffness to consider
%   - latAcc : [m/s^2] lateral acceleration to use for calculations
% Returns :
%   - k_chassis : [Nm/deg] vector for range of chassis stiffnesses considered
%   - dLLTD_by_dRSD :   |LLTD_r-LLTD_f| / |RSD_r - RSD_f|, where RSD is
%                       roll stiffness distribution as a percent and LLTD
%                       is lateral load transfer distribution as a percent
%                       averaged over RSD = 10-30% front and 70-90%
%                       front. This ignores the near linear portion of
%                       graph and the portion where |RSD_r - RSD_f| = 0
%   - LLT_front : [N] Load transferred from the front inside wheel to the 
%              front outside wheel. Rows vary roll stiffness distribution 
%              from 10-90% front and columns vary chassis stiffness from 
%              kChassis_min to kChassis_max

%   - LLT_rear : [N] Load transferred from the rear inside wheel to the 
%              rear outside wheel. Rows vary roll stiffness distribution 
%              from 10-90% front and columns vary chassis stiffness from 
%              kChassis_min to kChassis_max

% Based on models from:
%	Chalmers University Paper: http://publications.lib.chalmers.se/records/fulltext/191830/191830.pdf
%	SAE Paper : The Effect of Chassis Stiffness on Race Car Handling Balance. Deakin et. al.

kChassis_dataPoints = 1000; % number of points within the chassis stiffness range provided to use
kRoll_dataPoints = 160; % number of points to use for front and rear roll stiffness

g = 9.81; % [m/s^2] acceleration due to gravity
k_front = kRoll.*linspace(kRoll_pctFrontMin, kRoll_pctFrontMax,...
kRoll_dataPoints); % front suspension stiffness range [Nm/deg]
k_rear = kRoll.*linspace(1 - kRoll_pctFrontMin, 1 - kRoll_pctFrontMax,...
kRoll_dataPoints); % rear suspension stiffness range [Nm/deg]
k_chassis = linspace(kChassis_min, kChassis_max, kChassis_dataPoints); % range of chassis stiffnesses to consider [Nm/deg]
b = dist_cg_from_rear; % longitudinal distance from rear axle to CG [m]
a = wheelBase - b; % longitudinal position from front axel to CG [m]
x = hCG-(a*hRC_rear+b*hRC_front)/(wheelBase); % CG's Height over roll centre axis [m]

M_cg = mSprung*latAcc*x; % Momentum on roll axis from sprung mass [Nm]
% Moment from unsprung masses on front and rear axle
MUnsprung_front = latAcc*(rWheel_front-hRC_front)*mUnsprung_front; % [Nm]
MUnsprung_rear = latAcc*(rWheel_rear-hRC_rear)*mUnsprung_rear; % [Nm]

% Total cornering moment about roll axis, assume chassis stiffness is
% uniformly distributed and that the location of the CG will determine
% how much of the CG moment is transfered to the front and rear axle
M_front_total = MUnsprung_front + (b / (a + b)) * M_cg; % [Nm]
M_rear_total = MUnsprung_rear + (a / (a + b)) * M_cg; % [Nm]

LLT_front = zeros(kRoll_dataPoints, kChassis_dataPoints);
LLT_rear = zeros(kRoll_dataPoints, kChassis_dataPoints);
for iii = 1 : length(k_chassis)
	for jjj = 1 : length(k_front)
		%% Equation system
		% matrix relating angular deflection of suspension and chassis to 
		% moments and continuity equation
		A = [k_front(jjj) 0 -k_chassis(iii);
			0 k_rear(jjj) k_chassis(iii);
			1 -1 1];
		M_array = [M_front_total; M_rear_total; 0];

		angles_of_rotation = A \ M_array; % angles of rotation (front, rear, chassis) [deg]
		LLT_front(jjj, iii) = 1 / trackWidth_front * (k_chassis(iii) * angles_of_rotation(3) + M_front_total); % [N] Front left vertical wheel force - Front right vertical wheel force
		LLT_rear(jjj, iii) = 1 / trackWidth_rear * (-k_chassis(iii) * angles_of_rotation(3) + M_rear_total); % [N] Rear left vertical wheel force - Rear right vertical wheel force
	end
end

frontLoadDist = LLT_front./(LLT_front + LLT_rear);
frontRollStiffnessDist = k_front ./ (k_front + k_rear);

dLLTD_by_dRSD = zeros(1, length(k_chassis));
for iii = 1 : length(k_chassis)
	y = frontLoadDist(kRoll_dataPoints / 4 : 3 * kRoll_dataPoints / 4, iii)';
	x = frontRollStiffnessDist(kRoll_dataPoints / 4 : 3 * kRoll_dataPoints / 4);
	fit = polyfit(x, y, 1);
	dLLTD_by_dRSD(iii) = fit(1);
end
end
\end{lstlisting}
	
		\subsection{LLTD.m}
\begin{lstlisting}
clear all; close all
%% Lateral load transfer distribution for varying torsional stiffness
% Based on models from:
%	Chalmers University Paper: http://publications.lib.chalmers.se/records/fulltext/191830/191830.pdf
%	SAE Paper : The Effect of Chassis Stiffness on Race Car Handling Balance. Deakin et. al.

%% Lateral accelleration
latAcc = 2 * 9.81; %[m/s^2]
%% Car input static data [m]
% Car dimensions
hCG = 0.31122; % CG's height over ground contact line
rWheel_front = 9 * 25.4 / 1000; % Front wheel radius
rWheel_rear = 9 * 25.4 / 1000; % Rear wheel radius
hRC_front = 1.906 * 25.4 / 1000; % Front wheel roll centre height
hRC_rear = 2.234 * 25.4 / 1000; % Rear wheel roll centre height
trackWidth_front = 48 * 25.4 / 1000; % Track width
trackWidth_rear = 47 * 25.4 / 1000; % Track width
wheelBase = 60.5 * 25.4 / 1000; %[m]
rollStiffness =  850; %[Nm/deg]
% Masses
mSprung = 244; % Sprung mass, including driver [kg]
mUnsprung_front = 8; % Front unsprung mass [kg]
mUnsprung_rear = 7.5; % Rear unsprung mass [kg]
%% weight distributions to consider
b = wheelBase.*[0.6, 0.5, 0.4]; % CG's longitudal position from rear axle
% target percent of roll stiffness distribution difference that translates 
% into lateral load transfer distribution difference
tranferTarget = [0.8, 0.85, 0.9]; 
% set up colormap
col=jet(length(b) + 1);

% Generate curves for varrying dLLTD/dRSD vs k_chassis vs weight distribution
figure(1)
hold on

hInd = 1;
h = zeros(1,5);
hh = zeros(1,4);
for iii = 1 : length(b)
	[kChassis, dLLTD_by_dRSD, LLT_front, LLT_rear] = calc_lltd_per_rsd( mSprung, mUnsprung_front, mUnsprung_rear, hCG, b(iii), rWheel_front, rWheel_rear, hRC_front, hRC_rear, trackWidth_front, trackWidth_rear, wheelBase, rollStiffness, 0.1, 0.9, 1, 6000, latAcc);
	% diff_LLTD_per_RSD vs kChassis for this longitudinal position and a
	% line picking out the required chassis stiffness for 90% of 
	% rsd to translate to lltd
	figure(1)
	for jjj = 1 : length(tranferTarget)
		ind = dLLTD_by_dRSD - tranferTarget(jjj) > 0;
		ind = find(ind, 1, 'first');
		plot([0, kChassis(ind), kChassis(ind)], [dLLTD_by_dRSD(ind), dLLTD_by_dRSD(ind), 0], 'k')
	end
	rsX3 = rollStiffness * 3;
	rsX5 = rollStiffness * 5;
	index_rsX3 = interp1(kChassis, dLLTD_by_dRSD, rsX3);
	index_rsX5 = interp1(kChassis, dLLTD_by_dRSD, rsX5);
	
	if iii == 1
		h(hInd) = plot([0, rsX3, rsX3], [index_rsX3, index_rsX3, 0], '--', 'color', [1, 0.5, 0]);
		h(hInd + 1) = plot([0, rsX5, rsX5], [index_rsX5, index_rsX5, 0], 'r--');
		hInd = hInd + 2;
		text(rsX3, index_rsX3 - 0.025, ['\leftarrow (', num2str(rsX3), ', ', num2str(round(index_rsX3, 3)), ')'])
		text(rsX5, index_rsX5 - 0.025, ['\leftarrow (', num2str(rsX5), ', ', num2str(round(index_rsX5,3)), ')'])
	else
		plot([0, rsX3, rsX3], [index_rsX3, index_rsX3, 0], '--', 'color', [1, 0.5, 0]);
		plot([0, rsX5, rsX5], [index_rsX5, index_rsX5, 0], 'r--');
	end
	
	h(hInd) = plot(kChassis, dLLTD_by_dRSD, 'color',col(iii,:));
	hInd = hInd + 1;
	
	if iii == round(length(b) / 2)
		% plot LLTD % front VS RSD % front for a variety of chassis
		% stiffnesses 
		figure(2)
		hold on
		lower = length(LLT_front(:, 1)) / 4;
		upper = length(LLT_front(:, 1)) * 3 / 4;
		F_pctFront = LLT_front ./ (LLT_front + LLT_rear) * 100;
		hh(1) = plot(linspace(10, 90, length(LLT_front(:, 1))), F_pctFront(:, 16), 'color', col(1, :));
		fit = polyfit(linspace(30, 70, length(LLT_front(:, 1)) / 2 + 1), F_pctFront(lower : upper, 16)', 1);
		plot(linspace(30, 70, length(LLT_front(:, 1))), polyval(fit, linspace(30, 70, length(LLT_front(:, 1)))), 'k--');
		
		hh(2) = plot(linspace(10, 90, length(LLT_front(:, 1))), F_pctFront(:, 50), 'color', col(2, :));
		fit = polyfit(linspace(30, 70, length(LLT_front(:, 1)) / 2 + 1), F_pctFront(lower : upper, 50)', 1);
		plot(linspace(30, 70, length(LLT_front(:, 1))), polyval(fit, linspace(30, 70, length(LLT_front(:, 1)))), 'k--');
		
		hh(3) = plot(linspace(10, 90, length(LLT_front(:, 1))), F_pctFront(:, 200), 'color', col(3,:));
		fit = polyfit(linspace(30, 70, length(LLT_front(:, 1)) / 2 + 1), F_pctFront(lower : upper, 200)', 1);
		plot(linspace(30, 70, length(LLT_front(:, 1))), polyval(fit, linspace(30, 70, length(LLT_front(:, 1)))), 'k--');
		
		hh(4) = plot(linspace(10, 90, length(LLT_front(:, 1))), F_pctFront(:, 600), 'color', col(4, :));
		fit = polyfit(linspace(30, 70, length(LLT_front(:, 1)) / 2 + 1), F_pctFront(lower : upper, 600)', 1);
		plot(linspace(30, 70, length(LLT_front(:, 1))), polyval(fit, linspace(30, 70, length(LLT_front(:, 1)))), 'k--');
		
		figure(3)
		hold on
		xDim = size(LLT_front, 1);
		plot(linspace(1, 6000, length(LLT_front(1, :))), F_pctFront(xDim / 4, :), 'color', col(1, :))
		plot(linspace(1, 6000, length(LLT_front(1, :))), F_pctFront(xDim / 2, :), 'color', col(2, :))
		plot(linspace(1, 6000, length(LLT_front(1, :))), F_pctFront(3 * xDim / 4, :), 'color', col(3,:))
	end
end

figure(1)
xlabel('Chassis Stiffness [Nm/deg]')
ylabel('Region of Interest $\frac{\partial LLTF_{front}}{\partial RSD_{front}}$', 'Interpreter', 'Latex')
legend(h, {'kChassis = 3X roll stiffness', 'kChassis = 5X roll stiffness', ['Weight ' num2str(b(1) / wheelBase * 100) '% rear'], ['Weight ' num2str(b(2) / wheelBase * 100) '% rear'], ['Weight ' num2str(b(3) / wheelBase * 100) '% rear']}, 'Location', 'best')

figure(2)
legend(hh, {['k chassis = ' num2str(round(kChassis(16)))], ['k chassis = ' num2str(round(kChassis(50)))], ['k chassis = ' num2str(round(kChassis(200)))], ['k chassis = ' num2str(round(kChassis(600)))]}, 'Location', 'best');
xlabel('Stiffness Distribution % Front')
ylabel('LLTD % Front')

figure(3)
legend({'RSD % Front = 30', 'RSD % Front = 50', 'RSD % Front = 70'}, 'Location', 'best');
xlabel('Chassis Stiffness [Nm/deg]')
ylabel('LLTD % Front')
\end{lstlisting}